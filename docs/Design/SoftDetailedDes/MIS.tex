\documentclass[12pt, titlepage]{article}

\usepackage{amsmath, mathtools}

\usepackage[round]{natbib}
\usepackage{amsfonts}
\usepackage{amssymb}
\usepackage{graphicx}
\usepackage{colortbl}
\usepackage{xr}
\usepackage{hyperref}
\usepackage{longtable}
\usepackage{xfrac}
\usepackage{tabularx}
\usepackage{float}
\usepackage{siunitx}
\usepackage{booktabs}
\usepackage{multirow}
\usepackage[section]{placeins}
\usepackage{caption}
\usepackage{fullpage}

\hypersetup{
bookmarks=true,     % show bookmarks bar?
colorlinks=true,       % false: boxed links; true: colored links
linkcolor=red,          % color of internal links (change box color with linkbordercolor)
citecolor=blue,      % color of links to bibliography
filecolor=magenta,  % color of file links
urlcolor=cyan          % color of external links
}

\usepackage{array}

\externaldocument{../../SRS/SRS}

\input{../../Comments}
%% Common Parts

\newcommand{\progname}{Software Engineering} % PUT YOUR PROGRAM NAME HERE
\newcommand{\authname}{Team 17, Team RAdiAIdance
\\ Allison Cook
\\ Ibrahim Issa
\\ Mohaansh Pranjal
\\ Nathaniel Hu
\\Tushar Aggarwal} % AUTHOR NAMES

\usepackage{hyperref}
    \hypersetup{colorlinks=true, linkcolor=blue, citecolor=blue, filecolor=blue,
                urlcolor=blue, unicode=false}
    \urlstyle{same}
                                


\begin{document}

\title{Module Interface Specification for \progname{}}

\author{\authname}

\date{\today}

\maketitle

\pagenumbering{roman}

\section{Revision History}

\begin{tabularx}{\textwidth}{p{3cm}p{2cm}X}
\toprule {\bf Date} & {\bf Version} & {\bf Notes} \\
\midrule
01/11/2024 & 0.0 & Initial Document \\
01/12/2024 & 0.1 & Started adding MIS for Module sections \\
01/14/2024 & 0.2 & Added in Table from Module Guide (MG) into Module
  Decomposition section \\
01/16/2024 & 0.3 & Made small updates to Table 1 in Module Decomposition
  section; Continued adding MIS for Module sections \\
01/17/2024 & 0.4 & Completed the Symbols, Abbreviations and Acronyms and
  Introduction sections; Finished adding in MIS for Module sections \\
04/04/2024 & 0.5 & updated MIS for backend ML modules according to changes during implementation \\
\bottomrule
\end{tabularx}

~\newpage

\section{Symbols, Abbreviations and Acronyms}

This section records the symbols, abbreviations and acronyms information for
easy reference for terms used in this document. \\

For information on most of the symbols, abbreviations and acronyms referenced
in this document, see the SRS Documentation at the following link:
\url{https://github.com/tusharagg1/chest-x-ray-ai/blob/main/docs/SRS/SRS.pdf}. \\

The information on the rest of the symbols, abbreviations and acronyms
referenced in this document are shown in the table below. \\

\renewcommand{\arraystretch}{1.2}
\begin{tabular}{l l} 
  \toprule    
  \textbf{symbol} & \textbf{description} \\
  \midrule 
  AI/ML & Artificial Intelligence/Machine Learning \\
  \multirow{3}{*}{DICOM} & Digital Imaging and Communications in Medicine; \\
  & technical standard for digital storage/transmission \\
  & of medical images and related information \\
  GUI & Graphical User Interface \\
  \multirow{2}{*}{JPEG/JPG} & Joint Photographic Experts Group; digital image \\
  & compression standard, image format \\
  M & Module \\
  MG & Module Guide \\
  MVC & Model-View-Controller Software Architecture \\
  NLP & Natural Language Processing \\
  SRS & Software Requirements Specification \\
  \multirow{3}{*}{\progname} & The Process of Designing and Developing Software; \\
  & a reference to the software application described \\
  & in this document \\
  \bottomrule
\end{tabular} \\

\newpage

\tableofcontents

\newpage

\pagenumbering{arabic}

\section{Introduction}

The following document details the Module Interface Specifications for
the [Your Program Name Here] software application. This software application
(sometimes referred to as Software Engineering in this document) performs
scans of chest x-ray images, looking for diseases/infections and making
predictions. Those scan results and predictions of diseases/infections are
then put into natural language radiology reports (or components) and returned. \\

Complementary documents include the System Requirement Specifications and
Module Guide. The full documentation and implementation can be found at
\url{https://github.com/tusharagg1/chest-x-ray-ai/tree/main}.

\section{Notation}

%\wss{You should describe your notation.  You can use what is below as
%  a starting point.}

The structure of the MIS for modules comes from \citet{HoffmanAndStrooper1995},
with the addition that template modules have been adapted from
\cite{GhezziEtAl2003}.  The mathematical notation comes from Chapter 3 of
\citet{HoffmanAndStrooper1995}.  For instance, the symbol := is used for a
multiple assignment statement and conditional rules follow the form $(c_1
\Rightarrow r_1 | c_2 \Rightarrow r_2 | ... | c_n \Rightarrow r_n )$.

The following table summarizes the primitive data types used by \progname. 

\begin{center}
\renewcommand{\arraystretch}{1.2}
\noindent 
\begin{tabular}{l l p{7.5cm}} 
\toprule 
\textbf{Data Type} & \textbf{Notation} & \textbf{Description}\\ 
\midrule
character & char & a single symbol or digit\\
string & str & an array of characters \\
boolean & bool & True or False \\
integer & $\mathbb{Z}$ & a number without a fractional component in (-$\infty$, $\infty$) \\
natural number & $\mathbb{N}$ & a number without a fractional component in [1, $\infty$) \\
real & $\mathbb{R}$ & any number in (-$\infty$, $\infty$)\\
\bottomrule
\end{tabular} 
\end{center}

\noindent
The specification of \progname \ uses some derived data types: sequences, strings, and
tuples. Sequences are lists filled with elements of the same data type. Strings
are sequences of characters. Tuples contain a list of values, potentially of
different types. In addition, \progname \ uses functions, which
are defined by the data types of their inputs and outputs. Local functions are
described by giving their type signature followed by their specification.

\section{Module Decomposition}

The following table is taken directly from the Module Guide document for this project.

\begin{table}[H]
  \centering
  \begin{tabular}{p{0.3\textwidth} p{0.6\textwidth}}
    \toprule
    \textbf{Level 1} & \textbf{Level 2} \\
    \midrule

    {Hardware-Hiding Module} & MedInstInter \\
    \midrule

    \multirow{8}{0.3\textwidth}{Behaviour-Hiding Module} & ConvertDCM \\
    & ResultsGen \\
    & ChestXrayRead \\
    & ReportGen \\
    & DatabaseOps \\
    & UserAuthMgmt \\
    & AppGUI \\
    & Login \\ 
    & PerfScan \\
    & ViewResults \\
    & GenHeatmaps  \\
    & MLEndpoint  \\
    \midrule

    \multirow{5}{0.3\textwidth}{Software Decision Module} & AIModel \\
    & Backend \\
    & AppController \\
    \bottomrule

  \end{tabular}
  \caption{Module Hierarchy}
  \label{TblMH}
\end{table}
 
~\newpage

\section{MIS of Medical Institution Interface Module} \label{mMedInstInter}

\subsection{Module}
MedInstInter

\subsection{Uses}
N/A

\subsection{Syntax}

\subsubsection{Exported Constants}
N/A

\subsubsection{Exported Access Programs}

\begin{center}
  \begin{tabular}{p{3cm} p{4cm} p{4cm} p{5cm}}
    \hline
    \textbf{Name} & \textbf{In} & \textbf{Out} & \textbf{Exceptions} \\
    \hline
    connectToInst & instID: str, credentials: str & connectionStatus: bool &
      InvalidCredentialsException, InstNotFoundException \\
    \hline
  \end{tabular}
\end{center}

\subsection{Semantics}

\subsubsection{State Variables}
\begin{itemize}
    \item connectedInsts: Set(str) - maintains a set of connected institution IDs.
\end{itemize}
% \wss{Not all modules will have state variables.  State variables give the module
%   a memory.}

\subsubsection{Environment Variables}
\begin{itemize}
  \item InstsITSys: Set(str) - the set of external IT systems the application
    interfaces with to retrieve/exchange information.
\end{itemize}
% \wss{This section is not necessary for all modules.  Its purpose is to capture
%   when the module has external interaction with the environment, such as for a
%   device driver, screen interface, keyboard, file, etc.}

\subsubsection{Assumptions}
\begin{itemize}
  \item Patient data is stored in the medical institution's database, and the software intends to interface with their server to access that information.
\end{itemize}
% \wss{Try to minimize assumptions and anticipate programmer errors via
%   exceptions, but for practical purposes assumptions are sometimes appropriate.}

\subsubsection{Access Routine Semantics}

\noindent connectToInst():
\begin{itemize}
  \begin{item}
    transition:
    \begin{itemize}
      \item Adds `instID' to `connectedInsts' if the provided `credentials' is
        valid.
    \end{itemize}
  \end{item}
  \begin{item}
    output:
    \begin{itemize}
      \item `connectionStatus' is set to True if the connection is successful,
        False otherwise.
    \end{itemize}
  \end{item}
  \begin{item}
    exception:
    \begin{itemize}
      \item Throws `InvalidCredentialsException' if the provided credentials are invalid.
      \item Throws `InstNotFoundException' if the specified `instID' does not exist.
    \end{itemize}
  \end{item}
\end{itemize}

% \wss{A module without environment variables or state variables is unlikely to
%   have a state transition.  In this case a state transition can only occur if
%   the module is changing the state of another module.}

% \wss{Modules rarely have both a transition and an output.  In most cases you
%   will have one or the other.}

\subsubsection{Local Functions}
N/A
% \wss{As appropriate} \wss{These functions are for the purpose of specification.
%   They are not necessarily something that is going to be implemented
%   explicitly.  Even if they are implemented, they are not exported; they only
%   have local scope.}

\newpage

\section{MIS of Chest X-Ray Read Module} \label{mChXRR} 

\subsection{Module}
ChestXRayRead

\subsection{Uses}
MLEndpoint
% \begin{itemize}
 % \item Take chest X-ray as input in DICOM format, and extracts jpeg image to process in the model.
  %\item Process X-ray image using the ML model.
%\end{itemize}

\subsection{Syntax}

\subsubsection{Exported Constants}
N/A

\subsubsection{Exported Access Programs}

\begin{center}
  \begin{tabular}{p{3cm} p{4cm} p{4cm} p{5cm}}
    \hline
    \textbf{Name} & \textbf{In} & \textbf{Out} & \textbf{Exceptions} \\
    \hline
    initapp & None & bucket: firebase bucket & InvalidAuthException \\
    getDCMfiles & patientdir: directory for x-rays, bucket: firebase bucket & dcmfiles: array of DCM file data as byte-like objects & TypeException: invalid file format read \\
    \hline
  \end{tabular}
\end{center}

\subsection{Semantics}

\subsubsection{State Variables}
N/A

\subsubsection{Environment Variables}
N/A

\subsubsection{Assumptions}
Auth key for firbase authentication is stored as a seperate file that is accessed using its path.

\subsubsection{Access Routine Semantics}

\noindent initapp():
\begin{itemize}
  \begin{item}
    transition:
    \begin{itemize}
      \item Initiates the firebase bucket object to read chest x-ray files.
    \end{itemize}
  \end{item}
  \begin{item}
    output:
    \begin{itemize}
      \item A bucket object for the bucket storing the x-ray files for all patients.
    \end{itemize}
  \end{item}
  \begin{item}
    exception:
    \begin{itemize}
      \item Throws `InvalidAuthException' if authorization fails.
    \end{itemize}
  \end{item}
\end{itemize}

\noindent getDCMfiles():
\begin{itemize}
  \begin{item}
    transition:
    \begin{itemize}
      \item Reads DICOM files for a patient from the bucket to eventually pass to the ML model for processing.
    \end{itemize}
  \end{item}
  \begin{item}
    output:
    \begin{itemize}
      \item A list of x-ray images read from the DICOM files stored as bytes.
    \end{itemize}
  \end{item}
  \begin{item}
    exception:
    \begin{itemize}
      \item Throws `TypeException' if it reads a file of the wrong type.
    \end{itemize}
  \end{item}
\end{itemize}

\subsubsection{Local Functions}
N/A

\newpage

\section{MIS of Results Generation Module} \label{mResGen} 

\subsection{Module}
ResultsGen

\subsection{Uses}
\begin{itemize}
    \item AIModel
    \item GenHeatmaps
\end{itemize}

\subsection{Syntax}

\subsubsection{Exported Constants}
diseases: list of disease names that are checked

\subsubsection{Exported Access Programs}

\begin{center}
  \begin{tabular}{p{3cm} p{4cm} p{4cm} p{3cm}}
    \hline
    \textbf{Name} & \textbf{In} & \textbf{Out} & \textbf{Exceptions} \\
    \hline
    scanallxrays & procImgs: list of Raw Processed xray Image data & classification: Disease
      Classification & - \\
    \hline
  \end{tabular}
\end{center}

\subsection{Semantics}

\subsubsection{State Variables}
N/A

\subsubsection{Environment Variables}
N/A

\subsubsection{Assumptions}
\begin{itemize}
      \item The input list has one or more chest xray images.
      \item the images are stored as arrays of raw data (pixels in PNG format)
\end{itemize}

\subsubsection{Access Routine Semantics}

\noindent scanallxrays():
\begin{itemize}
  \begin{item}
    transition:
    \begin{itemize}
      \item generates prediction values for chest x-ray images. If more than one exists, the average value is taken for each disease. 
    \end{itemize}
  \end{item}
  \begin{item}
    output:
    \begin{itemize}
      \item 'classification' contains the generated disease prediction value for each disease.
    \end{itemize}
  \end{item}
  \begin{item}
    exception: N/A 
  \end{item}
\end{itemize}

\subsubsection{Local Functions}
\begin{itemize}
    \item getprediction: generates prediction values for a single image.
\end{itemize}

\newpage

\section{MIS of Report Generation Module} \label{mRepCompGen} 

\subsection{Module}
ReportGen

\subsection{Uses}
\begin{itemize}
    \item AIModel
\end{itemize}

\subsection{Syntax}

\subsubsection{Exported Constants}
N/A

\subsubsection{Exported Access Programs}

\begin{center}
    \begin{tabular}{p{3cm} p{3.5cm} p{5cm} p{3cm}}
    \hline
    \textbf{Name} & \textbf{In} & \textbf{Out} & \textbf{Exceptions} \\
    \hline
    generateReport & diagnosis: Disease Diagnosis & report: Radiology Report & - \\
    \hline
    \end{tabular}
\end{center}

\subsection{Semantics}

\subsubsection{State Variables}
N/A

\subsubsection{Environment Variables}
N/A

\subsubsection{Assumptions}
N/A

\subsubsection{Access Routine Semantics}

\noindent generateReport():
\begin{itemize}
\item transition: N/A 
\item output: \begin{itemize}
    \item 'report' contains the generated summary report based on the provided disease diagnosis.
\end{itemize}
\item exception: N/A
\end{itemize}

\subsubsection{Local Functions}
N/A

\newpage

\section{MIS of Database Operations Module} \label{Module} 

\subsection{Module}
DatabaseOps

\subsection{Uses}
N/A

\subsection{Syntax}

\subsubsection{Exported Constants}
N/A
\subsubsection{Exported Access Programs}

\begin{center}
\begin{tabular}{p{3cm} p{5cm} p{4cm} p{5cm}}
\hline
\textbf{Name} & \textbf{In} & \textbf{Out} & \textbf{Exceptions} \\
\hline
storeReport & report: Radiology Report, patientID: str & success: bool & ReportStorageException \\
retrieveReport & patientID: str & report: Radiology Report & ReportRetrievalException \\
\hline
\end{tabular}
\end{center}

\subsection{Semantics}

\subsubsection{State Variables}
\begin{itemize}
    \item 'connectDatabase: bool' indicates whether the module is currently connected to the database.
\end{itemize}

\subsubsection{Environment Variables}
N/A
\subsubsection{Assumptions}
N/A
\subsubsection{Access Routine Semantics}

\noindent storeReport():
\begin{itemize}
\item transition: \begin{itemize}
    \item Stores the provided 'report' in the database associated with the specified 'patientID'.
\end{itemize}
\item output: \begin{itemize}
    \item 'success' is set to True if the storing operation is successful, False otherwise.
\end{itemize}
\item exception: \begin{itemize}
    \item Throws 'ReportStorageException' if there is an issue storing the report.
\end{itemize}
\end{itemize}

\noindent retrieveReport():
\begin{itemize}
\item transition: \begin{itemize}
    \item Retrieves the radiology report associated with the specified 'patientID' from the database.
\end{itemize} 
\item output: \begin{itemize}
    \item 'report' contains the retrieved radiology report.
\end{itemize}
\item exception: \begin{itemize}
    \item Throws 'ReportRetrievalException' if there is an issue retrieving the report.
\end{itemize}
\end{itemize}

\subsubsection{Local Functions}
N/A
\newpage

\section{MIS of User Authentication/Management Module} \label{mUserAuthMgmt}
\subsection{Module}
UserAuthMgmt

\subsection{Uses}
N/A

\subsection{Syntax}

\subsubsection{Exported Constants}
N/A
\subsubsection{Exported Access Programs}

\begin{center}
\begin{tabular}{p{4cm} p{5cm} p{3cm} p{5cm}}
\hline
\textbf{Name} & \textbf{In} & \textbf{Out} & \textbf{Exceptions} \\
\hline
authenticateUser & username: str, password: str & status: bool & InvalidCredentialsException, UserNotFoundException \\
createUserAccount & username: str, password: str & success: bool & UserCreationException \\
deleteUserAccount & username: str, password: str & success: bool & UserDeletionException \\
checkAuthentication & username: str & isAuthorized: bool & - \\
\hline
\end{tabular}
\end{center}

\subsection{Semantics}

\subsubsection{State Variables}
N/A
\subsubsection{Environment Variables}
N/A
\subsubsection{Assumptions}
N/A
\subsubsection{Access Routine Semantics}

\noindent authenticateUser():
\begin{itemize}
\item transition: \begin{itemize}
    \item Verifies the provided 'username' and 'password' for authentication.
\end{itemize}
\item output: \begin{itemize}
    \item 'status' is set to True if authentication is successful, False otherwise.
\end{itemize} 
\item exception: \begin{itemize}
    \item Throws 'InvalidCredentialsException' if the provided credentials are invalid.
    \item Throws 'UserNotFoundException' if the specified user is not found.
\end{itemize}
\end{itemize}

\noindent createUserAccount():
\begin{itemize}
\item transition: \begin{itemize}
    \item Creates a user account with the provided 'username' and 'password'.
\end{itemize}
\item output: \begin{itemize}
    \item 'success' is set to True if the account creation is successful, False otherwise.
\end{itemize}
\item exception: \begin{itemize}
    \item Throws 'UserCreationException' if there is an issue creating the user account.
\end{itemize}
\end{itemize}

\noindent deleteUserAccount():
\begin{itemize}
\item transition: \begin{itemize}
    \item Deletes the user account associated with the specified 'username'.
\end{itemize}
\item output: \begin{itemize}
    \item 'success' is set to True if the account deletion is successful, False otherwise.
\end{itemize} 
\item exception: \begin{itemize}
    \item Throws 'UserDeletionException' if there is an issue deleting the user account.
\end{itemize}
\end{itemize}

\noindent checkAuthentication():
\begin{itemize}
\item transition: \begin{itemize}
    \item Checks whether the specified 'username' is currently authorized.
\end{itemize} 
\item output: \begin{itemize}
    \item 'isAuthorized' is set to True if the user is authorized, False otherwise.
\end{itemize}
\item exception: N/A
\end{itemize}

\subsubsection{Local Functions}
N/A

\newpage

\section{MIS of App GUI Module} \label{Module} 
\subsection{Module}
AppGUI
\subsection{Uses}
\begin{itemize}
    \item Login
    \item PerfScan
    \item ViewResults
\end{itemize}
\subsection{Syntax}

\subsubsection{Exported Constants}
N/A
\subsubsection{Exported Access Programs}

\begin{center}
\begin{tabular}{p{4cm} p{3cm} p{3cm} p{3cm}}
\hline
\textbf{Name} & \textbf{In} & \textbf{Out} & \textbf{Exceptions} \\
\hline
displayLoginPage & - & - & - \\
displayScanPage & - & - & - \\
displayResultsPage & - & - & - \\
\hline
\end{tabular}
\end{center}

\subsection{Semantics}

\subsubsection{State Variables}
N/A
\subsubsection{Environment Variables}
N/A
\subsubsection{Assumptions}
N/A
\subsubsection{Access Routine Semantics}

\noindent displayLoginPage():
\begin{itemize}
\item transition: Navigates to and displays the login page for the application.
\item output: N/A 
\item exception: N/A
\end{itemize}

\noindent displayScanPage():
\begin{itemize}
\item transition: Navigates to and displays the page for inputting an x-ray image for scanning.
\item output: N/A
\item exception: N/A
\end{itemize}

\noindent displayResultsPage():
\begin{itemize}
\item transition: Navigates to and displays the page for viewing scan results and reports.
\item output: N/A 
\item exception: N/A
\end{itemize}

\subsubsection{Local Functions}
N/A
\newpage

\section{MIS of Login Module} \label{Module} 
\subsection{Module}
Login
\subsection{Uses}
\begin{itemize}
    \item UserAuthMgmt
\end{itemize}
\subsection{Syntax}

\subsubsection{Exported Constants}
N/A
\subsubsection{Exported Access Programs}

\begin{center}
\begin{tabular}{p{2cm} p{5cm} p{4cm} p{5cm}}
\hline
\textbf{Name} & \textbf{In} & \textbf{Out} & \textbf{Exceptions} \\
\hline
login & username: str, password: str & loginStatus: bool & InvalidCredentialsException, UserNotFoundException \\
\hline
\end{tabular}
\end{center}

\subsection{Semantics}

\subsubsection{State Variables}
N/A
\subsubsection{Environment Variables}
N/A
\subsubsection{Assumptions}
N/A
\subsubsection{Access Routine Semantics}

\noindent login():
\begin{itemize}
\item transition: \begin{itemize}
    \item Authenticates the provided 'username' and 'password'.
\end{itemize}
\item output: \begin{itemize}
    \item 'loginStatus' is set to True if login is successful, False otherwise.
\end{itemize}
\item exception: \begin{itemize}
    \item Throws 'InvalidCredentialsException' if the provided credentials are invalid.
    \item Throws 'UserNotFoundException' if the specified user is not found.
\end{itemize}
\end{itemize}

\subsubsection{Local Functions}
N/A
\newpage

\section{MIS of Perform Scan Module} \label{Module} 
\subsection{Module}
PerfScan
\subsection{Uses}
N/A
\subsection{Syntax}

\subsubsection{Exported Constants}
N/A
\subsubsection{Exported Access Programs}

\begin{center}
\begin{tabular}{p{3cm} p{4cm} p{2cm} p{5cm}}
\hline
\textbf{Name} & \textbf{In} & \textbf{Out} & \textbf{Exceptions} \\
\hline
initiateScan & patient: selected patient & - & - \\
\hline
\end{tabular}
\end{center}

\subsection{Semantics}

\subsubsection{State Variables}
N/A
\subsubsection{Environment Variables}
\begin{itemize}
    \item patientID: $\mathbb{Z}$
\end{itemize}
\subsubsection{Assumptions}
N/A
\subsubsection{Access Routine Semantics}

\noindent initiateScan():
\begin{itemize}
\item transition: \begin{itemize}
    \item Receives the patient ID from the user to initiate the scanning process for that patient.
\end{itemize}
\item output: N/A 
\item exception: N/A
\end{itemize}

\subsubsection{Local Functions}
N/A
\newpage

\section{MIS of View Results Module} \label{Module} 
\subsection{Module}
ViewResults
\subsection{Uses}
\begin{itemize}
    \item PerfScan
\end{itemize}
\subsection{Syntax}

\subsubsection{Exported Constants}
N/A
\subsubsection{Exported Access Programs}

\begin{center}
\begin{tabular}{p{3cm} p{5cm} p{2cm} p{2cm}}
\hline
\textbf{Name} & \textbf{In} & \textbf{Out} & \textbf{Exceptions} \\
\hline
displayReport & report: Radiology Report & - & - \\
\hline
\end{tabular}
\end{center}

\subsection{Semantics}

\subsubsection{State Variables}
\begin{itemize}
    \item 'loading: bool' indicates if the backend has received the scan information to be displayed
\end{itemize}
\subsubsection{Environment Variables}
N/A
\subsubsection{Assumptions}
N/A
\subsubsection{Access Routine Semantics}

\noindent displayReport():
\begin{itemize}
\item transition: \begin{itemize}
    \item Displays the generated radiology report on the GUI.
\end{itemize}
\item output: N/A
\item exception: N/A
\end{itemize}

\subsubsection{Local Functions}
\noindent displayLoading():
\begin{itemize}
\item transition: \begin{itemize}
    \item Sets the display page to the loading screen, until loading is set to false
\end{itemize}
\item output: N/A
\item exception: N/A
\end{itemize}
\newpage

\section{MIS of AI Model Module} \label{Module} 
\subsection{Module}
AIModel
\subsection{Uses}
\begin{itemize}
    \item MLEndpoint
\end{itemize}
\subsection{Syntax}

\subsubsection{Exported Constants}
N/A
\subsubsection{Exported Access Programs}

\begin{center}
\begin{tabular}{p{3cm} p{4cm} p{4cm} p{5cm}}
\hline
\textbf{Name} & \textbf{In} & \textbf{Out} & \textbf{Exceptions} \\
\hline
getresponse & imgs: DICOM images & response: to send as JSON to be displayed & - \\
\hline
\end{tabular}
\end{center}

\subsection{Semantics}

\subsubsection{State Variables}
N/A
\subsubsection{Environment Variables}
N/A
\subsubsection{Assumptions}
N/A
\subsubsection{Access Routine Semantics}

\noindent getresponse():
\begin{itemize}
\item transition: \begin{itemize}
    \item uses the getresults, heatmapgen and reportgen modules to get predictions, heatmaps and report.
\end{itemize}
\item output: \begin{itemize}
    \item 'resp' contains the diagnostic results wrapped as a JSON object.
\end{itemize}
\item exception: N/A 
\end{itemize}

\subsubsection{Local Functions}
\begin{itemize}
    \item getrawimgs(): uses convertDCM module to convert DICOM image data to raw image pixel arrays.
\end{itemize}
\newpage

\section{MIS of Convert DCM Model Module} \label{Module} 
\subsection{Module}
convertDCM
\subsection{Uses}
\begin{itemize}
    \item AIModel
\end{itemize}
\subsection{Syntax}

\subsubsection{Exported Constants}
N/A
\subsubsection{Exported Access Programs}

\begin{center}
\begin{tabular}{p{3cm} p{4cm} p{4cm} p{5cm}}
\hline
\textbf{Name} & \textbf{In} & \textbf{Out} & \textbf{Exceptions} \\
\hline
getimgdata & DICOM file data & raw image (PNG) data & InvalidFormatException \\
getxraypngs & list of raw image (PNG) data & list of encoded xray images & InvalidFormatException \\
\hline
\end{tabular}
\end{center}

\subsection{Semantics}

\subsubsection{State Variables}
N/A
\subsubsection{Environment Variables}
N/A
\subsubsection{Assumptions}
N/A
\subsubsection{Access Routine Semantics}

\noindent getimgdata():
\begin{itemize}
\item transition: \begin{itemize}
    \item converts DICOM image data (bytes) to image array (PNG pixels).
\end{itemize}
\item output: \begin{itemize}
    \item raw image data as array of pixels.
\end{itemize}
\item exception: \begin{itemize}
    \item Throws 'InvalidFormatException' if the provided xray image is in an invalid format.
\end{itemize}
\end{itemize}

\noindent getxraypngs():
\begin{itemize}
\item transition: \begin{itemize}
    \item stores list of raw image data (pixels) as a list of encoded byte array of pixels in PNG format.
\end{itemize}
\item output: \begin{itemize}
    \item list of images as encoded array of bytes (PNG format).
\end{itemize}
\item exception: \begin{itemize}
    \item Throws 'InvalidFormatException' if the provided images are in an invalid format.
\end{itemize}
\end{itemize}

\subsubsection{Local Functions}
\begin{itemize}
    \item linstretchimg(): applies linear stretch in order to enhance the quality while converting DICOM files to a PNG format.
    \item getencodedimg(): used to convert a single image as raw pixel data to encoded array of bytes in PNG format.
\end{itemize}
\newpage

\section{MIS of Generate Heatmaps Module} \label{Module} 
\subsection{Module}
GenHeatmaps
\subsection{Uses}
\begin{itemize}
    \item AIModel
\end{itemize}
\subsection{Syntax}

\subsubsection{Exported Constants}
N/A
\subsubsection{Exported Access Programs}

\begin{center}
\begin{tabular}{p{3cm} p{4cm} p{4cm} p{5cm}}
\hline
\textbf{Name} & \textbf{In} & \textbf{Out} & \textbf{Exceptions} \\
\hline
genheatmappatient & List of Raw image data & heatmap image data & - \\
\hline
\end{tabular}
\end{center}

\subsection{Semantics}

\subsubsection{State Variables}
N/A
\subsubsection{Environment Variables}
N/A
\subsubsection{Assumptions}
N/A
\subsubsection{Access Routine Semantics}

\noindent genheatmappatient():
\begin{itemize}
\item transition: \begin{itemize}
    \item generates heatmaps for each disease and x-ray image for the patient
\end{itemize}
\item output: \begin{itemize}
    \item list of heatmap images stored as encoded array of bytes (PNG format).
\end{itemize}
\item exception: N/A
\end{itemize}

\subsubsection{Local Functions}
\begin{itemize}
    \item genheatmap(img): used to generate heatmaps for all diseases for a single xray image.
\end{itemize}
\newpage



\section{MIS of ML Endpoint Module} \label{Module} 
\subsection{Module}
MLEndpoint
\subsection{Uses}
N/A
\subsection{Syntax}

\subsubsection{Exported Constants}
N/A
\subsubsection{Exported Access Programs}

\begin{center}
\begin{tabular}{p{3cm} p{4cm} p{4cm} p{5cm}}
\hline
\textbf{Name} & \textbf{In} & \textbf{Out} & \textbf{Exceptions} \\
\hline
processfileendpoint & Directory name containing x-rays of the patient & response containing diagnosis, heatmaps and x-ray images to be displayed & InvalidRequestException \\
\hline
\end{tabular}
\end{center}

\subsection{Semantics}

\subsubsection{State Variables}
N/A
\subsubsection{Environment Variables}
N/A
\subsubsection{Assumptions}
N/A
\subsubsection{Access Routine Semantics}

\noindent processfileendpoint():
\begin{itemize}
\item transition: \begin{itemize}
    \item uses ChestXRayRead to read xray images and AIModel to generate response containing diagnosis data
\end{itemize}
\item output: \begin{itemize}
    \item response containing prediction values, heatmaps, summary report and x-ray image data for that patient
\end{itemize}
\item exception: \begin{itemize}
    \item throws 'InvalidRequestException' if the request has an invalid body or the xray files are not found.
\end{itemize}
\end{itemize}

\subsubsection{Local Functions}
\begin{itemize}
    \item mockresponse(): used for testing purposes to return a mock response to be sent to front end
\end{itemize}
\newpage


\section{MIS of Backend Module} \label{Module} 
\subsection{Module}
Backend
\subsection{Uses}
\begin{itemize}
    \item UserAuthMgmt
    \item MedInstInter
    \item DatabaseOps
\end{itemize}
\subsection{Syntax}

\subsubsection{Exported Constants}
N/A
\subsubsection{Exported Access Programs}

\begin{center}
\begin{tabular}{p{3cm} p{4cm} p{4cm} p{5cm}}
\hline
\textbf{Name} & \textbf{In} & \textbf{Out} & \textbf{Exceptions} \\
\hline
connectDatabase & credentials: str & connectionStatus: bool & InvalidCredentialsException \\
disconnectDatabase & - & success: bool & - \\
\hline
\end{tabular}
\end{center}

\subsection{Semantics}

\subsubsection{State Variables}

\subsubsection{Environment Variables}

\subsubsection{Assumptions}

\subsubsection{Access Routine Semantics}

\noindent connectDatabase():
\begin{itemize}
\item transition: N/A 
\item output: \begin{itemize}
    \item 'connectionStatus' is set to True if the connection is successful, False otherwise.
\end{itemize}
\item exception: \begin{itemize}
    \item Throws 'InvalidCredentialsException' if the provided credentials are invalid.
\end{itemize}
\end{itemize}

\noindent disconnectDatabase():
\begin{itemize}
\item transition: N/A 
\item output: \begin{itemize}
    \item 'success' is set to True if the disconnection is successful, False otherwise.
\end{itemize} 
\item exception: N/A 
\end{itemize}

\subsubsection{Local Functions}
N/A
\newpage

\section{MIS of App Controller Module} \label{Module} 
\subsection{Module}
AppController
\subsection{Uses}
\begin{itemize}
    \item AIModel
    \item NLPModel
    \item AppGUI
    \item Backend
\end{itemize}
\subsection{Syntax}

\subsubsection{Exported Constants}
N/A
\subsubsection{Exported Access Programs}

\begin{center}
\begin{tabular}{p{3cm} p{4cm} p{4cm} p{3cm}}
\hline
\textbf{Name} & \textbf{In} & \textbf{Out} & \textbf{Exceptions} \\
\hline
accessBackend & - & - & - \\
accessGUI & - & - & - \\
accessAI & - & - & - \\
accessNLP & - & - & - \\

\hline
\end{tabular}
\end{center}

\subsection{Semantics}

\subsubsection{State Variables}
N/A
\subsubsection{Environment Variables}
N/A
\subsubsection{Assumptions}
N/A
\subsubsection{Access Routine Semantics}

\noindent accessBackend():
\begin{itemize}
\item transition: \begin{itemize}
    \item Controller accesses the backend server.
\end{itemize}
\item output: N/A
\item exception: N/A
\end{itemize}

\noindent accessGUI():
\begin{itemize}
\item transition: \begin{itemize}
    \item Controller accesses the application GUI.
\end{itemize} 
\item output: N/A 
\item exception: N/A
\end{itemize}

\noindent accessAI():
\begin{itemize}
\item transition: \begin{itemize}
    \item Controller accesses the AI Model.
\end{itemize}
\item output: N/A
\item exception: N/A
\end{itemize}

\noindent accessNLP():
\begin{itemize}
\item transition: \begin{itemize}
    \item Controller acceses the NLP Model.
\end{itemize}
\item output: N/A 
\item exception: N/A
\end{itemize}

\subsubsection{Local Functions}
N/A

\newpage

\bibliographystyle {plainnat}
\bibliography {../../../refs/References}

\newpage

\section{Appendix} \label{Appendix}



\end{document}